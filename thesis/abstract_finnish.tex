

\thispagestyle{plain}
\begin{center}
    \Large
    \textbf{Koneellisen Keinomarkkinan Kehitt\"aminen}
        
    \vspace{0.4cm}
    \large
        
    \vspace{0.4cm}
    \textbf{Mikael Koli}\\
    % Strategic Finance and Business Analytics \\
    % Kauppatieteellinen tiedekunta \\
    % Lappeenrannan Teknillinen Yliopisto \\
    14.9.2020 % TODO
       
    \vspace{0.9cm}
    \textbf{Tiivistelmä}
\end{center}

% TODO: Purpose of the thesis is also to act as road map to build such a model
Koneelliset markkinat ovat kokeellisen taloustieteen alalaji, jonka tarkoituksena
on luoda keinotekoisesti sekä selittää todellisten markkinoiden dynamiikaa hyödyntäen
tietokonesimulaatioita. Empiirinen taloustieteen tutkijoilla on rajallinen kyky kontrolloida
markkinoiden sisäisiä ja ulkoisia muuttujia tutkiakseen ilmiöitä, mutta vastaavia rajoitteita
ei ole täysin laskennallisilla markkinoilla. Tämän Pro gradu -tutkielman tarkoituksena on rakentaa
geneerinen koneellinen markkina. Aiheen kirjallisuudesta puuttuu perinpohjaisia esimerkkejä siitä,
miten vastaavia koneellisia malleja luodaan, ja täten luodun mallin rakenne ja mekaniikka käydään
perusteellisesti läpi. Tässä tutkimuksessa luotu malli tähtää joustavuuteen ja sen tarkoituksena on
olla mahdollisimman geneerinen pitäen kompleksisuuden minimissä. Mallin kaupankävijät ovat budjettirajoitteisia
ja luonteeltaan nollaälyisiä. Mallissa luotu markkinat ovat muodoltaan jatkuvia
kaksisuuntaisia huutokauppoja. Malli tukee myös useita omaisuuslajeja sekä valuutan käsite on abstraktoitu: 
malli ei ennalta määrittele markkinoiden valuuttaa vaan tässä roolissa voi toimia mikä tahansa määritelty
omaisuuslaji. Mallin markkina validoidaan hyödyntäen tyyliteltyjä faktoja ja mallin dynamiikkaa
tutkitaan useilla kokeilla.

Malli tuottaa samankaltaisia tuloskia kuin vastaavat mallit aiemmasta kirjallisuudesta: markkinat nollaälyisillä
kaupankävijöillä konvertoituu tehokkaasti lähelle tasapainohintaa ja malli pystyy tuottamaan useita tyyliteltyjen 
faktojen ilmiöitä. Mallin kokeilu tuotti myös useita havaintoja. Osto- ja myyntitarjoukset ovat lähtökohtaisesti 
epäsymmetrisiä markkinoilla, joilla on määritelty hintainkrementti: ostotarjouksien limiittihintoja rajoittaa 
se, kuinka paljon ostajat pystyvät allokoimaan markkinoilla valuuttana toimivaa omaisuuslajia, mutta vastaavaa 
hintarajoitetta ei ole myyntitarjousten suhteen. Lisäksi simulaation omaisuuslajien tasapaino ei ole ainoa tekijä,
joka ajaa tasapainohintaa nollaälyisten kaupankävijöiden markkinoilla, vaan myös painot, joilla kaupankävijät 
asettavat osto- ja myyntitarjouksia, vaikuttaa tasapainohintaan.

\textbf{Avainsanat:} Laskennallinen markkina, Kokeellinen Taloustiede, Pörssisimulaattori