

\thispagestyle{plain}
\begin{center}
    \Large
    \textbf{Tietokonepohjaisen Keinomarkkinan Kehitt\"aminen}
        
    \vspace{0.4cm}
    %\large
        
    \vspace{0.4cm}
    \begin{flushleft}
        \normalsize
        \textbf{Tekijä:} Mikael Koli \\
        \textbf{Pääaine:} Strategic Finance and Business Analytics \\
        \textbf{Tiedekunta:} School of Business and Management \\
        \textbf{Pro Gradu -tutkielma:} Lappeenrantanan teknillinen yliopisto  \\
        \textbf{Vuosi:} 2020 \\
        \textbf{Tarkastajat:} Tutkijatohtori Jyrki Savolainen, professori Mikael Collan
    \end{flushleft}
       
    \vspace{0.5cm}
    \textbf{Tiivistelmä}
\end{center}

% TODO: Purpose of the thesis is also to act as road map to build such a model
Tietokonepohjaiset markkinat ovat kokeellisen taloustieteen alalaji, jonka tarkoituksena
on jäljentää ja selittää todellisten markkinoiden dynamiikaa hyödyntäen
tietokonesimulaatioita. Empiirisen taloustieteen tutkijoilla on rajallinen kyky kontrolloida
markkinoiden sisäisiä ja ulkoisia muuttujia ilmiöiden tutkimiseen, mutta vastaavia rajoitteita
ei ole täysin keinotekoisesti luoduilla markkinoilla. Tämän Pro gradu -tutkielman tarkoituksena on 
kehittää geneerinen tietokonepohjainen keinomarkkina. 

Mallin markkinat ovat rakenteeltaan kaksisuuntaisia huutokaupoja jatkuvalla tarjousten täsmäytyksellä.
Kaupankävijät ovat budjettirajoittuneita ja luonteiltaan minimiälyisiä. Malli tukee useita markkinoita, 
valuutan merkitys on abstraktoitu ja mikä tahansa omaisuuslaji voi toimia kaupankäyntivälineenä. 
Malli validoidaan hyödyntäen tyyliteltyjä faktoja ja mallin dynamiikkaa
tutkitaan useilla kokeilla.

Malli tuottaa samankaltaisia tuloskia kuin vastaavat mallit aiemmasta kirjallisuudesta: markkinat minimiälyisillä
kaupankävijöillä konvertoituu tehokkaasti lähelle tasapainohintaa ja malli pystyy tuottamaan useita tyyliteltyjen 
faktojen ilmiöitä. Mallin kokeilu tuotti useita lisähavaintoja, kuten että omaisuuslajien tasapaino ei ole ainoa tekijä,
joka ajaa tasapainohintaa minimiälyisten kaupankävijöiden markkinoilla, vaan myös painot, joilla kaupankävijät 
asettavat osto- ja myyntitarjouksia, vaikuttaa tasapainohintaan. Pelkästään tarjouskirjojen laitojen tasapaino vaikuttaa 
tasapainohintaan.

\textbf{Avainsanat:} Laskennallinen markkina, Kokeellinen Taloustiede, Pörssisimulaattori