
\section{Methodology}

\subsection{Julia Language}
Julia is relatively new dynamic programming language, 
reaching stable release of 1.0 in August 2018 \citep{JuliaV1}.
It is a performant language with an empahsis
on productivity. Even though Julia has focus on scientific 
computing it considered as a general purpose programming
language. Some of its features include
multiple dispatch, just-in-time compilation and built-in
matrix data types \citep{Julia}.

Julia fits well for building an artificial market.
Artificial markets can be seen as a set of objects
and interactions in between, as described \citet{Ben12},
and in such models object-oriented programming (OOP)
is required. Even though Julia is more functional, OOP
architecture can be constructed with use of Julia's 
structs and type annotated functions. Also, the 
speed of code execution, support for matrix operations 
and good productivity makes it viable language for 
prototyping simulation systems. 


\begin{lstlisting}
#= This is a code sample for the Julia language
(adapted from http://julialang.org) =#
struct Market
    
\end{lstlisting}