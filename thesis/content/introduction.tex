\section{Introduction}

% What's problematic with empirical research?
Due to the chaotic elements of the financial markets and lack of control, 
empirical research often face serious challenges in effort to study the 
dynamics of markets. The empirical researchers cannot control the market conditions 
and the thinking, intentions and actions of the market participants are often completely out of reach
for the researchers. The researchers may also encounter various practical challenges, such as 
lack of available data, spurious relations or changing market environment, 
which may furher limit their ability to produce robust and accurate results. 

% Solution for problems for empirics
However, the mechanics of how financial markets work, also known as market
microstructure, are well understood as they are inherently artificial structures 
even though the behaviour of the market participants is not. 
The mechanics of trading exchanges are simply sets of predefined rules 
that the exchanges follow and the market clearing, the procedure 
that matches the orders and produce trades, can be described with 
deterministic algorithms. Therefore the infrastructure for trading
could be accurately recreated and used to conduct laboratory 
experiments in which the unknowns could be reduced to the market participants
themselves. Such an experiment could produce complete information about
the individual orders, formed trades and positions of each trader at any given time
and the market conditions and information available for each market participants could
be controlled in order to minimize noise or to create a desired state of the market to
study. Moreover, the impact of the market microstructure itself to the trading prices and volumes 
is possible to extract and study. 

% Why artificial markets
These experiments often complement empirical research. They aim to find the
conditions in which empirically proven phenomena may form and close the gap
between theory and empirics. 

% Types of artificial markets
Such laboratory experiments are often referred as artificial markets, or artificial
stock market (ASM). \citet{boer05} discovered three categories of artificial markets: 
experimental, computational and analytical. In experimental artificial markets human traders
are the market participants and they most often trade using a software act as
the market system. Analytical markets are simply sets of mathematical equations that
form the market microstructure. Most often however, artificial markets are associated with 
computational artificial markets. These markets are computer simulations in which 
also the roles of market participants are performed by software agents. The focus in this
thesis is on the computational artificial markets and the term is used interchangeably
with the term \textit{artificial market}. 

% Generic overview of ASM
After the first computational artificial market developed in Santa-Fe Institute
(\citet{SantaFe94}) variety of models have been developed since. Some of the
models have advanced trading agents while others aim for minimal design 
regarding to the behaviour of the traders in order to study the market
mechanics. Some studies, such as \citet{GOYKHMAN20181729}, \citet{IZUMI200535},
\citet{Reinforcement09} and \citet{YEH20102089}, use learning algorithms in the decision making of 
their agents while some, such as \citet{God93}, \citet{Jam96}, \citet{Genoa01} and \citet{Raberto05}, 
aim for minimizing the intelligence of the traders to build a market settings in which
specific attributes emerge. % TODO: Continue with 

% Microstructure

\subsection{Motivation}

As discussed, there is fair amount of existing literature about artificial market models for various
purposes. However, the previous literature is lacking on generic studies about how to design
and build such models and therefore the main purpose of this thesis is to build a generic and abstract 
artificial market model with emphasis on the market microstructure. The design principles are to create flexible 
and well structured system to address possible further questions and ideas emerged after the project. 
The basis of the model is in the literature of market mechanics and in auction literature. The choice for the
market type was influenced by the most dominant market mechanics of real stock markets and therefore
continuous double auction, or limit order book market, was chosen as the microstructure. As the 
role of the market participants for this thesis' purpose is in little importance, they are designed
simplistic and with as few assumptions as practical. Therefore, the traders are chosen to act randomly
with limited intelligence. Such traders are also known as zero-intelligent traders. 

Second purpose of this thesis is to complement the literature of artificial markets to study 
the behaviour of the created market and discuss the dynamics formed by the market and reasons for them.
The well known and empirically proven characteristics of real financial market dynamics, known as 
stylized facts, are used to study how well the model resembles the dynamics of real markets.

Third purpose of this thesis is to demonstrate the capabilities of Julia programming language.
Julia language is relatively new general purpose language but the main focus of the language is 
on scientific computing.


\subsection{Structure of the Thesis}
% TODO: There may be additional part if artificial market is left
The thesis consists of multiple parts. First, the market microstructure including the theoretical 
background of the real financial markets, price dynamics, clearing mechanics and stylized facts 
are explained and after this, some general principles and components of artificial markets are 
discussed. Then relevant examples of computational artificial markets from the literature
are presented. 

The empirical section of this thesis is divided into three sections. First, the mechanics and
structure of the created artificial market model are described and then the model is tested
with several scenarios. Finally conclusions of the study are presented and the limitations and
possible further research topics are discussed.