\section{Introduction}

% What's problematic with empirical research?
Due to the chaotic elements of the financial markets and lack of control, 
empirical research in finance often faces serious challenges in its effort to study the 
market dynamics. The market conditions cannot be controlled by the researchers
and the thinking, intentions and actions of the market participants are often impossible to study. 
The researchers may also encounter various practical challenges, such as 
lack of available data, spurious relations or changing market environment, 
which may further limit their ability to produce credible, robust and accurate results. 

% Solution for problems for empirics
However, the platform where trading takes place, the market
microstructure, is well understood as they are inherently artificial structures. 
The mechanics of exchanges are simply sets of predefined rules 
that the exchanges follow, and the market-clearing, the procedure 
that matches the orders and produces trades, can be described with 
deterministic algorithms. Therefore, the infrastructure for trading
could be accurately recreated and used to conduct laboratory 
experiments in which the unknowns could be reduced to the market participants
themselves. Such an experiment produces complete information about
the individual orders, formed trades and positions of each trader at any given time
and the market conditions and information available for each market participants could
be controlled to minimize noise or to create a desired state of the market to
study. Moreover, the impact of the market microstructure itself on the market prices and trading 
volumes is possible to extract and study. Such experiments often complement the research of empirical finance.
They aim to find the conditions in which empirically proven phenomena may form and 
bring theory and empirics closer together. 


% Types of artificial markets
Such laboratory experiments are often referred to as artificial markets or artificial
stock market (ASM). \citet{boer05} discovered three categories of artificial markets: 
experimental, computational and analytical. In experimental artificial markets, human traders
are the market participants and they most often trade using software acting as
the market system. Analytical markets are simply sets of mathematical equations that
form the market microstructure. Most often, however, artificial markets are associated with 
computational artificial markets. These markets are computer simulations in which 
also the roles of the market participants are performed by software agents. The focus in this
thesis is on the computational artificial markets and the term is used interchangeably
with the term \textit{artificial market}. 

% Generic overview of ASM
After the first computational artificial market developed in Santa-Fe Institute
(\citet{SantaFe94}) variety of models have been developed since. Some of the
models have advanced trading agents while others aim for minimal design 
regarding the behaviour of the traders to study the market
mechanics. Some studies, such as \citet{GOYKHMAN20181729}, \citet{IZUMI200535},
\citet{Reinforcement09} and \citet{YEH20102089}, use learning algorithms in the decision making of 
their agents while some, for example \citet{God93}, \citet{Jam96}, \citet{Genoa01} and \citet{Raberto05}, 
aim for minimizing the intelligence of the traders to build a market setting in which
specific characteristics emerge. % TODO: Continue with 

% TODO: What this thesis is about


\subsection{Motivation}

As mentioned, there is a fair amount of existing literature about artificial market models for various
purposes. However, the previous literature is lacking on generic studies on how to design
and build such models and therefore the main purpose of this thesis is to build a generic and abstract 
artificial market model with an emphasis on the market microstructure. The design principles are to create a flexible 
and well-structured system to serve as a platform for future studies in the area of artificial markets. 
The basis of the model is in the literature of market microstructure and in auction literature. The choice for the
market type was influenced by the most dominant market mechanics of real stock markets and therefore
continuous double auction, or limit order book market, was chosen as the clearing mechanics. As the 
role of the market participants for this thesis' purpose is in little importance, they are designed
simplistic and with as few assumptions as practical. Therefore, the traders are chosen to act randomly
with limited intelligence. Such traders are also known as zero-intelligent traders. 

The second purpose of this thesis is to complement the literature of computational artificial markets by studying 
the behaviour of price dynamics in such a model.
The well-known and empirically proven characteristics of real financial market dynamics are used to 
study how well the dynamics of the model resembles the dynamics of the real markets.

The research questions the thesis aims to answer are as follows:
% Tutkimuskysymys: "miten rakentaa geneerinen malli?" 
\begin{enumerate}
    \item What are the aspects needed to be taken into consideration in designing of a computational 
          artificial market according to the existing scientific literature?
	\item How to program a computational artificial market be constructed that is both abstract and generic?
    \item How does the market price behave in a computational artificial market with continuous
          double auction as the market microstructure and zero-intelligent traders as the 
          trading agents?
\end{enumerate}


\subsection{Structure of the Thesis}
% TODO: There may be additional part if artificial market is left
This thesis consists of several parts. First, the market microstructures which are in place in real 
markets are discussed. This section includes a short theoretical background of auctions, the execution and clearing mechanics in place in real
stock markets and empirically proven price dynamics observed in markets known as stylized
facts. Then some generic issues specific to artificial markets are discussed including the time handling and creation
of trading agents. Then some relevant examples of artificial market models from the literature are discussed. 

The empirical section of this thesis is divided into three sections. First, the mechanics and
structure of the created artificial market model are described. Then the model is tested
with several experiments. Finally, the conclusions of the study are presented and the limitations of the model and
possible further research topics are discussed.