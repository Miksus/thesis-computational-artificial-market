\section{Computational Artificial Market}
% What is ASM, what kinds of structures they have
Artificial markets are simulation environments for analyzing, 
reproducing and gaining understanding of the dynamics of 
real markets \citep{Julien07}. However, there are several 
characteristics in real markets that cannot be replicated
in artificial setting in an obvious way. Some of these artificial
market specific issues, such as how the meaning of time is 
implemented and how the decision making of the trading agents
works, are presented in this section.


\subsection{Time Handling in ASM}

Implementing the meaning of time to an ASM is a non-trivial problem.
One may do the simple assumption that the trades occur in fixed
intervals in stock markets. This is however rarely the case in real markets
and the issue of nonsynchronous trading, also known as nontrading, emerges. 
For instance, an asset may be traded in minute basis when there is
public information released while other times it may not be traded much at all.
There may even be days without any trade. \citep{Econometrics} 

In ASM literature time handling is typically divided into two implementations:
synchronous and asynchronous. \citet{Julien07} summarized the difference as that in synchronous
trading all the traders are given the same opportunity to make their placements
while in asynchronous trading the simulation engine gives the traders the opportunity to
make decisions in disorganized manner, possibly by picking traders randomly.
\citet{Ben17} argued the difference to be more about discrete versus continuous time handling.
According to them, synchronous trading takes place in fixed time intervals whereas in 
asynchronous trading the order submission can occur in any point in time. Unarguably, 
asynchronous trading is more realistic as time is in fact continuous and traders make
their decisions in their individual frequencies.

% TODO: How this goes in this thesis?

\subsection{Trading Agents}
\label{section:ASMTradingAgents}
% TODO: Human traders?

There are multitude types of trading agents used in ASM models with varying degree
of rationality. Some models use as simplistic agents as possible to implement as
little complexity to the model as possible while some try to emulate rational 
markets with more advanced but complex traders. The most simplistic trading agents
used in the literature are traders that submit random orders. Such traders
lack the ability to do informed and rational decisions hence they are often called
zero-intelligent traders. On the other end of the spectrum, there are traders
that use artificial intelligence (AI) in their decision making. The models using
more advanced trading agents are often referred as \emph{agent-based computational
economics} (ACE).

% ZI traders
Term zero-intelligent (ZI) traders was initially introduced by \citet{God93}. Zero-intelligent
traders do not aim for maximizing profit or any other function and they do not 
possess the abilities to remember or to learn. Zero-intelligent traders are also unable
to manage risks, balance portfolio or speculate. The limit prices they submit are
often drawn either from a uniform distribution with a specified range (\citet{God93}, \citet{Mil08}) 
or from normal distribution with a mean of the last market price or the
best price of opposite book (\citet{LIU20082535}, \citet{Genoa01}). ASM models using
zero-intelligent traders are discussed with more detail in literature review. 

% ACE
\citet{LeBranon2000} argumented that the design decision on more advanced traders 
lies in whether the traders should be evolving or learning. However, the literature
generally does not make clear distinction between the terms and the term \emph{learning}
is often used to describe also evolving models. An extensive review of ACE literature 
is presented by \citet{ACE12}.

% Evolutionary
In evolutionary models the agents converge to the optimal strategy using evolutionary algorithms, typically
genetic algorithms or genetic programming. Evolutionary models are characterized by 
mutation and selection mechanics: sets of parameters for the model are created 
using random mutation between the best performing sets of parameters from last 
round, ie. trading session. The worst performing sets of parameters are discarded.
\citet{GenAlgASM18} and \citet{GenAlgBTCASM19} used genetic algorithm to select 
best performing strategies from a set of predefined trading rules. The predefined
rules were combined to form strategies using random mutation. \citet{FieldGA05} developed
similar model except instead of rules defined by the researchers they derived the rules 
by interviewing actual traders. \citet{LeBranon2001} combined an evolutionary model with a 
learning algorithm: the trading rules used in the evolutionary model were formed using a 
simple feed-forward neural network.

% Learning
Learning ASM models use machine learning for the agents to adopt the optimal strategies.
However, such models are somewhat rarer in the literature than evolutionary
models. \citet{Reinforcement09} used reinforcement 
learning, more specifically Q-learning, to predict the fundamental value of a 
dividend paying stock and this prediction was used in the agents' 
decision making: if the predicted fundamental value is smaller than the market price
the trader will sell the stock and if it is higher the trader will buy it. They 
also tested the model with an evolutionary selection to replace worst performing
traders with best performing.

% Another example of evolutionary: \citet{CHEN2001363} 
% There seems to be no pure machine learning agents even though topic mentioned in 
% LeBranon2000 and Chen S. et al (2012) Agent-based economic models and econometrics
% Used search words: 
%   Agent based computational economics & (machine learning | neural network)
%   artificial stock market & (machine learning | neural network)

% TODO: Heterogenous vs Homogenous traders

ACE models are generally specific and their goal is often to simulate a specific behaviour.
They also introduce higher amount of design decisions to validate and model parameters 
to tune compared to zero-intelligent agents. Due to these factors zero-intelligent traders
were chosen as the trader type in this thesis. The goal of the introduced model is to be as 
generic as possible and the focus is on the market microstructure and therefore the more 
simplistic approach is more suited.