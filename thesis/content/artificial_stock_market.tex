\section{Computational Artificial Market}
% What is ASM, what kinds of structures they have
Artificial markets are simulation environments for analyzing, 
reproducing and gaining understanding of the dynamics of the 
real markets \citet{Julien07}. 

Based on the market mechanics, \citet{boer05} distinguished 
three types of artificial markets: experinental, computational and analytical. 
Experimental models are controlled laboratories 
where trading is conducted by humans whereas in computational
models it is done by software agents. Analytical models are a
set of mathematical equations for describing the market
mechanics. \citet{boer05} noted that 

\subsection{Time Handling in ASM}

Implementing the meaning of time to an ASM is a non-trivial problem.
One may do the simple assumption that the trades occur in fixed
intervals in stock markets. This is however rarely the case in real markets
and the issue of nonsynchronous trading, also known as nontrading, emerges. 
For instance, an asset may be traded in minute basis when there is
public information released while other times it may not be traded much.
There may even be days without any trade. \citep{Econometrics} 

In ASM literature time handling is typically divided into two implementations:
synchronous and asynchronous. \citet{Julien07} summarized the difference as that in synchronous
trading all the traders are given the same opportunity to make their placements
while in asynchronous trading the simulation engine gives the traders the opportunity to
make decisions in disorganized manner, possibly by picking traders randomly.
\citet{Ben17} argued the issue to be more about discrete versus continuous time handling.
According to him, synchronous trading takes place in fixed time intervals whereas in 
asynchronous trading an order submission can occur in any point in time. Unarguably, 
asynchronous trading is more realistic as the time is continuous and traders make
their decisions in their individual frequencies.

Nonsynchronous trading phenomenon may emerge also in synchronous

\subsection{Trading Agents}

There are multitude types of trading agents used in ASM models with varying degree
of rationality. Some models use as simplistic agents as possible to implement as
little complexity to the model as possible while some try to emulate rational 
markets with more advanced but complex traders. The most simplistic trading agents
used in the literature are traders that submit random orders. Such traders
lack the ability to do informed and rational decisions hence they are often called
zero-intelligent traders. On the other end of the spectrum, there are traders
that learn and evolve. 

% ZI traders
Term zero-intelligent traders was initially used by \citet{God93}. Zero-intelligent
traders do not aim for maximizing profit or any other function and they do not 
possess the abilities to remember or learn. The limit prices they submit are
often drawn either from a uniform distribution with a specified range (\citet{God93}, \citet{Mil08}) 
or from normal distribution with a mean of the last market price or the
best price of opposite book (\citet{Genoa01}, \citet{LIU20082535}).

% Learning & evolving


% Heterogenous vs Homogenous traders