\section{Literature Review}

Santa Fe institute's computational artificial stock 
market is considered to be one of the first of a kind. 
The focus of the simulation structure is more in the 
decision making process of the traders rather than in the 
market's microstructure. The model's traders use 
genetic algorithm to form a pool of rules to follow
in their decision making process. The traders can buy or
sell either risk free bond or risky dividend yielding stock.
The market is quote-driven in a sense that the traders
either accept the current price set up by the clearing house
or do not trade. The price goes up in case of there is excess
demand and vice versa. (\citeauthor{SantaFe94}, \citeyear{SantaFe94} 
and \citeauthor{SantaFe99}, \citeyear{SantaFe99})

Since then there have been many more models being 
developed with various complexities. These models are often developed
for achieving specific properties of a real market, 
usually related to the price formation. In this section
go through recent literature to describe some of the models
that has been built and discuss the design decisions in them.


\citep{Ben12}

\subsection{Zero Intelligent Markets}

\subsubsection{Gode's and Sunder's Market}
\citet{God93} observed market efficiency in a market setting
of minimal intelligence. They conducted a partly human and partly
computational laboratory experiment to replicate the market
efficiency of human traders with least intelligent computational
trading agents. \\

Their market is organized as a double auction with several
specifications. The market is continuous in a sense that the 
orders are continuously matched even though the trading was
conducted in speicified sessions. The traders were divided 
into buyers and sellers. Buyers profits were tied to their 
individual redemption values that are gained at the end of
each trading session for each bough unit and sellers' profits
were tied to their individual cost of unit. Redemption values
represent the demand function of each buyer and cost of units 
the supply function of each seller for each session. The
traders were allowed to submit orders with size of one unit
and the trade price was taken from the price stated in the 
first of the crossing orders. \\

The computational trading agents were zero intelligent traders 
who trade at random price from a predefined range. 
The first simulation was run with budget constrain: only the bids
under the redemption value and the asks over the cost of unit 
were allowed. The second simulation was run without such limitation.
These simulations were compared with the data from human traders. \\

\citeauthor{God93} found that the budget constrain is enough to 
achieve market efficiency very close to human traders. Their findings
suggest that intelligence does not to play critical role in terms of 
how to build an efficient artificial market. The more important aspect
is the characteristics of the market, not the traders. \\


\subsubsection{Jamal's and Sunder's Market}
% Continuation study from Jamal & Sunder
\citet{Jam96} expanded \citeauthor{God93}'s work and added behavioural 
elements to similar market structure. They tested the rationality of
a market with three different trader populations: bayesian, empirical
bayesian and heuristical traders. Bayesian and empirical bayesian acted
as baselines for the market rationality and the heuristical traders were
the treatment populations for which the researchers added elements of behavioural
biases: representativeness bias and anchor-and-adjust. \\

The traders in all of the populations were zero intelligent in terms of the
prices they bid and ask. Most notable difference to \citet{God93}'s market
is that instead of redemption value and cost of units determining the
supply and demand values, \citeauthor{Jam96} used private dividend payouts 
to act as such. The dividend payout is random but traders are informed
about two possible dividends. There is also an imperfect signal that the
traders can utilize to determine which of the payouts is more probable. \\

\citeauthor{Jam96} concluded that the rationality of their markets were
not caused by the rationality of the investors but the market structure.
The implemented beharioural biases in the heuristic traders were not enough 
to steer the double auction market away from the bayesian equilibrium.
They highlight the role of the market microstructure in the rationality 
of the artificial markets. \\

\subsubsection{Raberto's and Cincotti's Market}
% Another example of Intelligence vs Structure
\citet{Raberto05} experimented with ZI investors in market setting
where each computational traders can submit bid and ask offers. Their
market was also a continuous double auction which has realistic order
book setup. In their market, the traders issue random bid and ask limit
orders with a price that follows normal distribution. The mean of this 
normal distribution is the best price of the opposite book and the 
standard deviation is constant in the first experiment and linearly 
dependent on the volatility of the stock in the second experiment. The
traders do have a budget constrain. \\

\citeauthor{Raberto05} found that the fat tailed returns observed in
real stock markets can be achieved in artificial settings using limit 
order book in the price formation and it is not consequence of clearing 
house mechanism itself. They also found that volatility clusters can
be achieved with a simple dependency on the volatility. Their paper
is furher evidence that the market microstructure may be more
important than the complexity of the traders in effort to replicate
many of the properties of real markets.

% Counter Argument for usefullness of ZI
\subsubsection{Counter Argument against ZI markets}
\citet{Mil08} provided counter argument on the usefullness of zero
intelligent traders. They replicated market structure from
\citeauthor{God93}'s study with one major difference: how the 
time is handled. In their market version, the demand varies
also between sessions and buyers can plan for it by buying units
in advance for periods of higher demand. Unlike human traders,
zero intelligent traders obviously cannot plan ahead and a market
populated with such traders cannot gain allocative efficiency comparable
to a market populated with human traders. 


\subsection{Generic ASM Frameworks}

Most of the literature in the field of artificial stock markets 
focus on recreating phenomena from real markets in artificial settings
and often these studies are lacking in details of how the model was created.
They may include descriptions how the clearing is done or how the trading agents 
come up with their decisions but usually they do not discuss about practical matters 
such as how the interactions between the simulation components work, where the 
information about traders' positions are stored or how the simulation is orchestrated. 
Studies that describe how to actually build a realistic computational 
market are quite rare. In this section discuss about some of the generic
frameworks designed in the field in order to illustrate how to structure the computational
environment.


\subsubsection{Object Oriented ASM}
% Gaia
\citet{Ben12} developed a realistic but abstract artificial stock market
model for continuous double auction using a modeling methodology known as 
Gaia. Gaia is a software design methodology for agent-based systems which 
highlights identifying the roles of the system and interactions in between.
The final model the researchers came up with is an object oriented 
approach consisting of five roles: a market, traders, a company, banks
and external world. The roles of the market and the traders in the model 
are rather similar as discussed already in this thesis. Interestingly,
even though banks and companies are identified as separate agents, 
the amount of cash and stocks are tracked in the trader agents themselves.
They, however, do not have permission to update their wealth independently. 
Banks and companies, on the other hand, act as middlewares for informing 
traders when and how much to update the amount cash and stocks. Banks also pay 
fixed interest based on the amount of cash each trader holds and companies 
pay dividend for each share. External world is a container of exogenous 
information, such as paid dividends or news, that can be used by the traders for 
their decision making. The researchers also discussed that the model could be
simplfied to only trading agents and market agent by including banks', companies' 
and extrnal world's roles to the trader agents in case of less complex but 
less detailed model is required.

The orders in their model requires the following information to be valid:
stock name, time-stamp, volume and the quoted price. These orders are stored
in the market's order book, in case they cannot be immediately matched, but 
also in the trader agents themselves for possible cancelling purposes.

% Something about the interaction model and order matching

As they did not build a working software using the model, it is unclear
how practical the model is and how easy it is to program. The model, 
however, is detailed and served as an inspiration in this thesis even though
its structure and design decisions were not strictly followed.


\subsubsection{ASM with a Concrete Implementation}
% Testing double auction as a component within a generic market model architecture
\citet{Julien07} built a simpler abstract representation of an ASM consisting
only of the trading agents, market and external world. On top of this model, 
they also created a more detailed concrete framework which they tested with
zero intelligent traders and asynchronous continuous double auction as market
model. The framework is modular in a way that it allows heterogenous traders
that emit and accept different amount of information to live in the same 
simulation environment. The researchers used translators in between the 
trading agents and the market in order to transform the orders submitted by
the trader agents and the information emitted by the market to a unified format.
For example, in case a submitted order had only the direction and the quantity 
defined, the translator fulfills the order price with the current market price 
transforming the order essentially as market order. The framework also had a
unit called simulation engine to handle the interactions between the components.

The flow of the system consisted of six steps which can be summarized with three:
\begin{enumerate}
    \item The simulation engine let all or some of the trading agents submit their orders 
            using the exogenous information provided by the external world and 
            information about the balance between supply and demand provided by the market.
    \item The simulation engine commands the market to clear.
    \item The simulation engine commands external world to update its information.
\end{enumerate}


\begin{figure}
    \includegraphics[width=\linewidth]{diagrams/julien_market.png}
    \caption{Framework defined by Derveeuw et al. (2007)}
    %\label{fig:boat1}
\end{figure}

% Something to wrap up like where the positions are saved?