\section{Literature Review}

Santa Fe institute's computational artificial stock %TODO Supervisor: add year of creation. DONE!
market (\citeyear{SantaFe94}) is often considered to be the first of a kind. 
The focus of the model is more in the 
decision-making process of the traders rather than in the 
market's microstructure. The model's traders use a 
genetic algorithm to form a pool of rules to base
their trading decisions. The traders can buy or
sell either a risk-free bond or a risky dividend-yielding stock.
The market is quote-driven in the sense that the traders
either accept the current price given by the matching engine
or do not trade. The price moves up in case if there is excess
demand and vice versa. (\citeauthor{SantaFe94}, \citeyear{SantaFe94} 
and \citeauthor{SantaFe99}, \citeyear{SantaFe99})

Since then, there have been many models being 
developed with various complexities. These models are often developed
for achieving specific properties of a real market, 
usually related to price formation. In this section
the recent literature is discussed to describe some of the models 
that have been built and the design decisions in them are discussed.

%\citep{Ben12}

\subsection{Zero Intelligent Markets}

% What are zero intelligence traders?

The zero-intelligent trading agents were presented in the section 
~\ref{section:ASMTradingAgents}. In this chapter, some literature concerning 
models using such trading agents is presented and discussed with greater detail.
% TODO: Extend. When such agents are used?

\subsubsection{Gode's and Sunder's Market}
\citet{God93} observed market efficiency in a market setting
of minimal intelligence. They conducted a partly human and partly
a computational laboratory experiment in an effort to replicate the market
efficiency of human traders with the least intelligent computational
trading agents. 

The market \citet{God93} created is organized as a double auction with several
specifications. The market is continuous in the sense that the 
orders are continuously matched even though the trading was
conducted in specified sessions. The traders were divided 
into buyers and sellers. Buyers' profits were tied to their 
individual redemption values that were gained at the end of
each trading session for each unit bought and sellers' profits
were tied to their individual cost of unit. Redemption values
represent the demand function of each buyer and the cost of units 
the supply function of each seller for each session. The
traders were allowed to submit orders with a size of one unit
and the trade price was drawn from the price stated in the 
first of the crossing orders.

The computational trading agents of \citet{God93}'s market were zero-intelligent traders 
who trade at random prices from a predefined range. 
The first simulation was run with a budget constrain: only the bids
under the redemption value and the asks over the cost of unit 
were allowed. The second simulation was run without such limitation.
These simulations were compared with the data from human traders
in similar trading environment.

\citet{God93} found that the budget constrain is enough to 
achieve market efficiency very close to human traders. Their findings
suggest that intelligence does not play a critical role in terms of 
how to build an efficient artificial market. The more important aspect
is the characteristics of the market, not the characteristics of the traders.

\subsubsection{Jamal's and Sunder's Market}
% Continuation study from Jamal & Sunder
\citet{Jam96} expanded \citeauthor{God93}'s work and added behavioural 
elements to a similar market structure. They tested the rationality of
a market with three different trader populations: Bayesian, empirical
Bayesian and heuristical traders. Bayesian and empirical Bayesian acted
as baselines for the market rationality and the heuristic traders were
the treatment populations for which the researchers added elements of behavioural
biases: representativeness bias and anchor-and-adjust. 

The traders in all of the populations were zero-intelligent in terms of the
prices they bid and ask. Most notable difference to \citet{God93}'s market
is that instead of redemption value and cost of units determining the
supply and demand values, \citeauthor{Jam96} used private dividend payouts 
to act as such. The dividend payout is random but traders are informed
about two possible dividends. There is also an imperfect signal that the
traders can utilize to determine which of the payouts is more probable. 

\citet{Jam96} concluded that the rationality of their markets was
not caused by the rationality of the investors but the market structure.
The implemented behavioural biases in the heuristic traders were not enough 
to steer the double auction market away from the bayesian equilibrium.
They highlight the role of the market microstructure in the rationality 
of the artificial markets. 


% Counter Argument for usefulness of ZI
\subsubsection{Counter Argument against ZI markets}
\citet{Mil08} provided a counter-argument on the usefulness of zero
intelligent traders. He replicated the market structure from
\citet{God93}'s study with one major difference: how the 
time is handled. In his version of the market, the demand varies
also between sessions and buyers can plan for it by buying units
in advance for periods of higher demand. Unlike human traders,
zero-intelligent traders obviously could not plan ahead and a market
populated with such traders did not gain allocative efficiency comparable
to a market populated with human traders. Therefore, when time is a 
meaningful element, as by having changing market conditions, zero-intelligent 
trading agents are insufficient. However, \citet{Mil08} also suggested
that in a market where the conditions change in a predefined way, as was in 
his experiment, a slight enhancement to the mechanics of the zero-intelligent 
agents may be sufficient to overcome this limitation.



\subsection{Generic ASM Frameworks}

Most of the literature in the field of artificial stock markets 
focus on recreating phenomena from real markets in artificial settings.
The details of the algorithms used in the market microstructure or in
the decision making of the trading agents are often discussed with limited 
depth. Studies may include descriptions of how the clearing is done or how the trading agents 
come up with their decisions but they usually do not discuss practical matters 
such as how the interactions between the simulation components work, where the 
information about traders' positions are stored or how the simulation is orchestrated. 
Studies that describe how to build a realistic computational 
market are relatively rare but not completely absent. In this section, some of the 
generic frameworks designed in the field are discussed to illustrate how 
to structure a computational environment.


\subsubsection{Object Oriented ASM}
% Gaia
\citet{Ben12} developed a realistic but abstract artificial stock market 
framework for continuous double auction using a modelling methodology known as 
Gaia. Gaia is a software design methodology for agent-based systems which 
highlights identifying the roles of the system and interactions in between. 
\citet{Ben12} came up with an object-oriented framework consisting of three sets of submodels: 
an agent model, a service model and an acquaintance model. The agent model is consisting of 
five roles: a market, traders, a company, banks and the external world. 
The roles of the market and the traders in the model are rather similar 
as discussed already in this thesis. Interestingly, even though banks and 
companies are identified as separate agents, the amount of cash and stocks 
are tracked in the trading agents themselves. The trading agents, however, do not have permission 
to update their wealth independently. Banks and companies, on the other hand, act 
as a middleware for informing traders when and how much to update the amount of cash 
and stocks. Banks also pay fixed interest based on the amount of cash each trader 
holds and companies pay dividend for each share. The external world is a container of 
exogenous information, such as paid dividends or news, that can be used by the 
traders for their decision making. The researchers also discussed that the model 
could be simplified to only trading agents and market agent by including banks', 
companies' and external world's roles to the trader agents in case of the less complex but 
less detailed model is required.

The second model \citet{Ben12} designed, the service model, 
consists of the functionalities of the agents. This model is simply the list of 
functions or services, each agent has and the inputs, outputs, preconditions and postconditions 
in them. For example, a trading agent has a service \textit{generate order} which takes 
expected price, expected dividend per share, the number of shares owned by the trader, 
the amount of cash the trader has, market state and information about the external world as input 
and the service outputs an order. The postcondition of this method is that the output order 
must not be non-existent. The final model, the acquaintance model, is simply the interactions 
between the agents. This model consists of the links of how information is passed around the 
agents but not the content of it.

% Something about the interaction model and order matching

As \citet{Ben12} did not build a working software using the framework, it is unclear 
how practical the model is and how convenient it is to program. The framework, 
however, is detailed and extensive and it served as an inspiration in this thesis even though 
its structure and design ideas were not strictly followed. 


\subsubsection{ASM with a Concrete Implementation}
% Testing double auction as a component within a generic market model architecture
\citet{Julien07} built a simpler abstract representation of an ASM consisting 
only of the trading agents, market and external world. On top of this model, 
they also created a more detailed concrete framework which they tested with 
zero-intelligent traders and with an asynchronous continuous double auction as the market 
microstructure. The framework is modular in a way that it allows heterogeneous traders 
that emit and accept a different amount of information to live in the same 
simulation environment. The researchers used components called translators in between the 
trading agents and the market in order to transform the orders submitted by
the trading agents and the information emitted by the market to a unified format.
For example, if a submitted order had only the direction and the quantity 
defined, the translator fulfils the order price with the current market price 
transforming the order essentially to a market order. The framework also had a
unit called simulation engine to handle the interactions between the components.

The flow of the system can be summarized with three steps:
\begin{enumerate}
    \item The simulation engine let all or some of the trading agents submit their orders. 
            The placement decision can utilize the exogenous information provided by the 
            external world and information about the balance between supply and demand 
            provided by the market.
    \item The simulation engine commands the market to clear.
    \item The simulation engine commands the external world to update its information.
\end{enumerate}


\begin{figure}[H]
    \includegraphics[width=\linewidth]{diagrams/julien_market.png}
    \caption{Framework defined by \citet{Julien07}}
    %\label{fig:boat1}
\end{figure}

% Something to wrap up like where the positions are saved?

\subsection{Reproduction of Stylized Facts with ASM}
As discussed before, stylized facts are especially useful
to validate the representativeness of ASM models. There are several 
studies that aim to replicate these phenomena in 
artificial settings. Some of these studies are
gone through in this section.
% Some generic stuff about these studies
% 
\subsubsection{Genoa Market}

\citet{Genoa01} created a double auction market with zero-intelligent 
traders. Their used call market as their market microstructure. The traders draw their 
order prices from a normal distribution in which the mean price is 10 \%
higher for bids and 10 \% lower for asks from the last market price. The
rationale for this is to encourage the traders to submit orders that more likely get 
fulfilled increasing the liquidity of the market.
They also implemented volatility feedback: the standard deviation for 
the distribution used for drawing the order prices is derived from the 
historical volatility of the asset and the time window for which this
volatility is calculated is independent for each trader. In addition,
they also introduced networking between the traders: the traders
form clusters by randomly pairing with each other. Clusters are activated
randomly and when one is activated, all of the traders belonging to it
submit same sided orders, either bids or asks. After the activation, the cluster is destroyed.

Using this architecture \citet{Genoa01} were able to produce some of the stylized facts
including fat-tailed returns and volatility clusters. However, the also discussed
that the volatility clusters are prone to increasing the size of the model and
that the produced volatility clusters are exponentially decaying while empirics have
shown them to be a power-law decaying.

\subsubsection{Limit Order Market}
% Another example of Intelligence vs Structure
\citet{Raberto05} experimented with zero-intelligent investors in a market setting
in which each computational trader can submit bid and ask offers. They
implemented a continuous double auction market with a realistic order
book setup. In their market, the traders issue random bid and ask limit
orders with prices drawn from a normal distribution. The mean of this 
normal distribution is the best price of the opposite book and the 
standard deviation is constant in the first experiment and linearly 
dependent on the volatility of the stock in the second experiment. The
traders also are budget constrained. 

\citet{Raberto05} found that the fat-tailed returns observed in
real stock markets can be achieved in artificial settings using limit 
order book as the price formation and it is not a consequence of the clearing 
house mechanism itself. They also found that volatility clusters can
be achieved with a simple dependency on the volatility, and thus the paper
provides evidence that the market microstructure may be more
important than the complexity of the traders in an effort to replicate
many of the properties of real markets.
%TODO: what did this mean: "it is not consequence of the clearing house mechanism itself"?

\subsubsection{Market Clearing And Minority Game}

\citet{LIU20082535} studied stylized facts with two sets of experiments: with
a market populated by zero-intelligent traders and with a market formed using 
a model called \textit{Minority Game}. The first experiment focused on studying
the market microstructure by observing the impact of clearing frequency for the 
presence of stylized facts in a call market. The traders were not restricted with budget 
constrain and they traded using order prices drawn from a normal distribution 
in which the mean and standard deviation were set to constants. The reason 
for these decisions was that they were not interested in the converge to 
the equilibrium thus the fair value of the stock was not important. They 
observed that the autocorrelation diminishes and fat-tailed returns emerge 
when the frequency of clearing the call market is increased. Moreover,
it was found out that these phenomena are caused by the market microstructure
as the increase in clearing frequency decrease the density of orders and 
increase the price spread. % TODO Supervisor: Too many "they" DONE!

\citet{LIU20082535}'s second experiment was more specific. The intention of the experiment 
was to observe the effect of the behavioural elements using a model called
Minority Game. Minority Game is a simple decision-making system in
which each trader has a set of behavioural rules. From these rules, each
trader picks the one that has the highest score in terms of predicting 
historical outcomes of the prices. If a rule did not predict correctly,
its score is decreased. With this experiment, they found out that the
behavioural elements can also cause fat-tailed returns, cross-correlation 
of returns and volume and absence of autocorrelation.


\subsubsection{Summary of Relevant Literature}

The literature regarding zero intelligent traders often focuses 
on observing microstructure's impact on price efficiency or reproduction 
of stylized facts. While markets populated with zero intelligent traders 
have observed to be efficient and able to produce several of the stylized 
facts they lack in ability to come up with efficient prices in changing 
market conditions. According to the literature, the market microstructure 
may be in key position regarding reproduction of price dynamics observed 
in real markets. 

There are some examples of generic artificial market models in the literature. 
The core components of such a market are the market itself and the trading agents 
and they are share functional and object-oriented elements. The main interactions 
are the order placement in which a trader informs the market about its desire to 
buy or sell and the relay of information from the market to the traders. The 
literature is summarized in the table ~\ref{tbl:literature_summary}.


\newcolumntype{L}[1]{>{\raggedright\let\newline\\\arraybackslash\hspace{0pt}}m{#1}}

\begin{flushleft}
    \begin{table}[htbp]
        %\centering
        %\setlength\tabcolsep{1pt}
        \scriptsize
        \renewcommand{\arraystretch}{1.2}
        
        \caption{Summary of Relevant Literature}
        \label{tbl:literature_summary}
        
        \begin{tabularx}{\textwidth}{lcL{3.2cm}L{3.2cm}L{3.2cm}}
            \toprule
            \textbf{Author}    &   \textbf{year}    &   \textbf{Market type}    &   \textbf{Investor type}    &   \textbf{Research target} \\
            \midrule
            Gode \& Sunder         &   1993    &   Continuous double auction    &   Zero intelligent                                    &   Allocative efficiency of ZI traders \\
            \hline
            Jamal \& Sunder        &   1996    &   Continuous double auction    &   Bayesian, empirical bayesian \& biased heuristic    &   Rationality of markets \\
            \hline
            Miller                 &   2008    &   Continuous double auction    &   Zero intelligent                                    &   Counter argument on usefulness of ZI traders \\
            \hline
            Benhammada \& Chikhi   &   2012    &   Unspecified                  &   Unspecified                                         &   Create a generic architecture \\
            \hline
            Derveeuw et al.        &   2007    &   Continuous double auction    &   Zero intelligent                                    &   Create a generic model and model stylized facts \\
            \hline
            Raberto et al.         &   2001    &   Call market                  &   Zero intelligent with trading networks              &   Model stylized facts \\
            \hline
            Raberto \& Cincotti    &   2005    &   Continuous double auction    &   Zero intelligent with volatility feedback           &   Model stylized facts \\
            \hline
            Liu et al.             &   2008    &   Call market                  &   Zero intelligent                                    &   Model stylized facts in perspective clearing frequency \\
            \bottomrule
        \end{tabularx}
    \end{table}
\end{flushleft} % TODO Supervisor: order by the year. DONE!

