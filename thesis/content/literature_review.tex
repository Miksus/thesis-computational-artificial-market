\section{Literature Review}

Santa Fe institute's computational artificial stock 
market is considered to be one of the first of a kind. 
The focus of the simulation structure is more in the 
decision making process of the traders rather than in the 
market's microstructure. The model's traders use 
genetic algorithm to form a pool of rules to follow
in their decision making process. The traders can buy or
sell either risk free bond or risky dividend yielding stock.
The market is quote-driven in a sense that the traders
either accept the current price set up by the clearing house
or do not trade. The price goes up in case of there is excess
demand and vice versa. (\citeauthor{SantaFe94}, \citeyear{SantaFe94} 
and \citeauthor{SantaFe99}, \citeyear{SantaFe99})

Since then there have been many more models being 
developed with various complexities. These models are often developed
for achieving specific properties of a real market, 
usually related to the price formation. In this section
we take a look recent literature to describe some of the models
that has been built and the design decisions in them.


\citep{Ben12}