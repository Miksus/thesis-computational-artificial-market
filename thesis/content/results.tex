
\section{Results}

% TODO Supervisor: Can you put a table listing the experiments? Columns: experiment, market setting, simulation runs, purpose etc. DONE!
% if these can be put clearly, the text can tell (cannot read)
The model was tested in several experiments. First, the generic dynamics of 
the model is inspected. The convergence to equilibrium price is studied with two
simulations and then the model is validated using stylized facts. 
The purpose of the first set of experiments is to validate the model and provide additional evidence 
on the observations from previous literature regarding to efficiency of zero-intelligent traders and 
reproduction of stylized facts.

The second set of experiments consists of examining price dynamics by injecting shocks to the order book via placement 
preferences of the traders. This is more experimental in nature and the purpose is to gain a better understanding of 
how the market price moves due to the structure of the order book in the model. The summary of the experiments is 
shown in the table ~\ref{tbl:experiment_summary}.


\begin{table}[ht]
    \centering
    
    \caption{Summary of the Experiments}
    \label{tbl:experiment_summary}

    \begin{adjustbox}{width=1\textwidth}
    \begin{tabular}{lllllp{4cm}}
        \toprule
        {} & Sessions & Starting & Probability & Probability &                                            Study target \\
        \textbf{Experiment             } &  &    price      &       of bid          &       of ask          &                                                    \\
        \midrule
        \textbf{Lower start } &     200  &           500  &                33\% &                33\% &                Convergence to equilibrium \\
        \midrule
        \textbf{Higher start} &     200  &         1 500  &                33\% &                33\% &                Convergence to equilibrium \\
        \midrule
        \textbf{At equilibrium         } &   4 000  &         1 000  &                33\% &                33\% &  Presence of Stylized facts \\
        \midrule
        \textbf{Positive shock         } &     200  &         1 000  &                43\% &                23\% &  Price dynamics of imbalanced order book \\
        \midrule
        \textbf{Negative shock         } &     200  &         1 000  &                23\% &                43\% &  Price dynamics of imbalanced order book \\
        \bottomrule
    
    \end{tabular}

\end{adjustbox}
\end{table} 


In all the experiments all the traders are allowed 
to make one trading decision per session, meaning that the speak ratio 
is one. There are no dividend nor interest paid for any assets in these experiments.
The summary of the parameters that are set as the same in all of the experiments 
is shown in the table ~\ref{tbl:parameter_summary}.
% TODO:
%   Why equilibrium testing? Literature has inspected it and this is to closer this model to it
%   Why stylized fact validation? Literature uses these to validate
%   Why shock injection? Because of something new.

\begin{table}[H]
    \centering
    
    \caption{Summary of Shared Parameters Between the Experiments}
    \label{tbl:parameter_summary}

    \begin{adjustbox}{width=1\textwidth}
    \begin{tabular}{llp{6cm}}
        \toprule
        %{}   & & \\
        \textbf{All experiments              } & &             \\
        \midrule
        Number of traders                      &      10 000 & \\
        Number of markets                      &           1 & \\
        Number of assets                       &           2 & \\
        \midrule
        Initial amount of currency per trader  &  10 000 000 & \\
        Initial number of stocks per trader    &       1 000 & \\
        \midrule
        Speak ratio                            &       100\% & \\
        Probability of inaction                &        33\% & \\
        Std Dev of the price draw              &          20 & \\
        \bottomrule
    \end{tabular}

\end{adjustbox}
\end{table} 

\subsection{Generic dynamics of the model}
% Price dynamics, stylized facts, order book evolution
The single asset simulation was run three times: for the first run
the starting market price was set to 500, for the second run it
was set to 1 500 and for the third was set to 1 000 which is near the equilibrium price. 
The rationale for this equilibrium price is discussed later.
The first and second simulations were
used to study how the convergence to the equilibrium price occurs
and the third to examine the stylized facts of the price
dynamics after the equilibrium is reached. The simulations were
run with 10 000 zero-intelligent traders which is the 
same amount of traders as what \citet{Raberto05} used in their experiments
with somewhat similar settings. As only the convergence to equilibrium is 
the point of interest in the first and second simulations, the number of sessions
were chosen to be relatively small, 200 sessions, but to get reliable results for
the third simulation the number of sessions was chosen to be 4 000 in order to 
make sure the statistical significance of the tests performed to validate the 
stylized facts will not be hampered by the number of observations.

The amount of currency and the amount of stock 
a trader owns at the beginning of all the simulations were set to
10 000 000 units and 10 000 units respectively. Therefore
the ratio of stock to currency is 1 000 which
is the expected equilibrium price as there are
no payouts in either asset. The amount of currency
per trader may seem much but as the tick size is one,
it may be more convenient to think the currency as amounts
of cents or pennies. The standard deviation of
the normal distribution that the traders use to pick the order prices
is set to 20. % TODO Supervisor: See if you can put some quantities to the table (see upper). DON'T remove explanations. DONE!

%\begin{table}
%    \input{tables/simulation_summary.tex}% table.tex > \mytable
%\end{table}


\subsubsection{Converge to Equilibrium}
% Descriptive analysis:
%   how converge occurs
The evolution of trade prices, quantities and trade values throughout
the trading sessions for the lower and higher starting price simulations 
are shown in the figure ~\ref{fig:basic_trades}. As can be observed,
the near-equilibrium market price is achieved before 30 trading
sessions for both experiments. The simulation with lower starting price converges
to the equilibrium in almost instantly whereas for the higher starting
price it takes more sessions. Furthermore, the traded quantities are
initially lower for the simulation with higher staring price than 
for the simulation with lower price. As the traded volumes are lower,
the converge occurs slower due to that there are fewer situations in
which the market price can adapt. In addition to these findings, 
the equilibrium price is slightly lower than the expected which is 
about 940.
%The reason why the price stays slightly 
%under the equilibrium price and why the price converges quicker with
%lower starting price than with the higher may be caused by the implementation of
%determining the order price and order quantity. If a trader decides
%to allocate less currency on an order than the order price it chose 
%the order gets obviously invalidated but there is no such
%limitation for placing asks. Therefore, there might be a slightly less
%bids than asks in general.

\begin{figure}[H]
    \includegraphics[width=\linewidth]{plots/basic_trades.png}
    \caption{Evolution of price and quantity} % TODO Supervisor: "in the simulation one using... showing" Explaing hypotheses and the plot more. Same with plots 6.2-6.4. DONE!
    \label{fig:basic_trades}
\end{figure}

% How the market converged to equilibrium
The market depths, as shown in figure ~\ref{fig:market_depths}, suggest that the
converge to equilibrium is caused by balancing the sides of the market. 
In this figure, the order book is visualized for the first 15 trading sessions
for both of the simulations: with a starting market price of 500 and 1 500. 
The surface describes the evolution of the cumulative bids and cumulative asks 
throughout time and the bottom of the valley formed in the surface is the bid-ask spread. 
The left side from the valley is the cumulative bid book and right is the cumulative ask book. 
The ask side of the market is almost completely missing initially in the run that
starts with lower market price while the opposite is true for the run with higher
initial market price. The ask side of the order book emerges almost instantly 
in the simulation of the lower starting price whereas the converge happens more 
gradually with a higher starting price.


\begin{figure}[H]
    \centering
    \begin{subfigure}{.5\textwidth}
      \centering
      \includegraphics[width=\linewidth]{plots/basic_market_depth_converge_lower.png}
      \caption{Starting price 500}
      \label{fig:market_depth_lower}
    \end{subfigure}%
    \begin{subfigure}{.5\textwidth}
      \centering
      \includegraphics[width=\linewidth]{plots/basic_market_depth_converge_higher.png}
      \caption{Starting price 1 500}
      \label{fig:market_depth_higher}
    \end{subfigure}
    \caption{Converge of the order book to equilibrium}
    \label{fig:market_depths}
\end{figure}
% TODO Supervisor: remove extra enters. DONE!
Figure ~\ref{fig:basic_orderbook_evo} provides more top-down view of the orderbook.
This figure shows mostly the same aspects as figure ~\ref{fig:market_depths} but the sides of the 
order book are not extrapolated to the edges of the plot: the white colour represents the absence of orders,
the green colour the cumulative bids and the red colour the cumulative asks.
The figure suggests that the asking traders in the experiment with the lower initial market price
adopt the equilibrium price quicker than the bidding traders in the experiment with the higher initial market 
price. One possible explanation for this phenomenon might be the bid-ask asymmetricity
discussed earlier. It is also noteworthy that the downward spikes found in the figure when the equilibrium
is reached are caused by that some traders had consumed most of their currency balances and still decided 
to do a bid order. Because of the budget constrain, these traders had to submit bid orders with prices
they could afford and these prices were substantially lower than the market price and may seem like outliers.

% TODO: Mention that the reason for high price - low price differing convergence to equilibrium is inspected in shocks

\begin{figure}[H]
    \includegraphics[width=\linewidth]{plots/basic_order_book_evo.png}
    \caption{Order book evolution}
    \label{fig:basic_orderbook_evo}
\end{figure}

When the equilibrium is reached, the order book is rather steep as is illustrated in the figure
~\ref{fig:basic_orderbook_evo}. This figure consists of the last 50 sessions of the extended simulation.
The spread also is also relatively stable most likely due to the number of traders. 

\begin{figure}[H]
    \includegraphics[width=\linewidth]{plots/basic_market_depth_in_equilibrium.png}
    \caption{Order book depth in equilibrium}
    \label{fig:basic_orderbook_evo}
\end{figure}

% TODO: additional finding about explaining the next price

\subsubsection{Model Validation with Stylized Facts}
% stylized facts
% Something about filtering fist 100 out and now inspecting stylized facts
As stated before, the representativeness of the model's price behaviour compared to real markets 
is studied using stylized facts. In order to have representative stationary data set to 
test the stylized facts, the sessions before reaching the price equilibrium are filtered out.
Removing the first 100 sessions was considered suitable for this purpose leaving total of 3900 
sessions to study the stylized facts. 
As already discussed, the longer simulation that starts near-equilibrium 
price is used to study the presence of stylized facts in the model. 

% Autocorrelation
% TODO: REWRITE
The autocorrelation of the at equilibrium experiment is inspected using autocorrelation function (ACF) and partial 
autocorrelation function (PACF). ACF and PACF functions for the intrasession absolute prices and absolute returns 
are plotted in the figure ~\ref{fig:basic_autocorr}. Intrasession in this context is defined as the period 
between each trade. It should be noted that this interval is non-fixed and may have the effects of non-trading. 
The x-axes of the plots represent the lags and y-axes of the plots represent the 
autocorrelation coefficients. The coefficient for lag zero is always one as a value is always completely autocorrelated with itself. 
The absolute returns in this context are the actual price movements of 
the asset and absolute values are simply the actual prices of the asset. As the prices are already in equilibrium, and therefore should be 
stationary, taking the third difference from the absolute value, calculating the rate of returns, should not be necessary. There is some negative 
autoregressive (AR) process in the absolute returns as the ACF plot shows negative and slowly decaying lags. There are numerous 
significant lags in the AR process as can be observed from the PACF plot. In addition, the prices themselves are 
positively autocorrelated with an AR process as the ACF shows slowly decaying lags. These findings are in line with the experiments conducted by \citet{Raberto05}: 
they also observed some negative autocorrelation in intraday returns and long-lasting autocorrelation in the absolute
values with their simulation. However, it should be noted that this figure shows very short term
autocorrelations and the time between the lags may not be fixed: one lag is simply derived from one
trade. 
% https://www.youtube.com/watch?v=R-oWTWdS1Jg


\begin{figure}[H]
    \includegraphics[width=\linewidth]{plots/basic_autocorrelation_intra.png}
    \caption{Autocorrelation of the simulation}
    \label{fig:basic_autocorr}
\end{figure}

Figure ~\ref{fig:basic_autocorr_per_session} suggests that the autocorrelation is carried over one
session for the absolute returns and four sessions in terms of prices. The x-axis and y-axis of the figure are the same as 
in figure ~\ref{fig:basic_autocorr}: x-axis is the lag number and y-axis is the autocorrelation coefficient. 
The prices used in this figure are the last trade prices of each session. As can be interpreted from the figure, there is an AR(1) process in 
the absolute returns per session and AR(4) for prices on a session basis.

\begin{figure}[H]
    \includegraphics[width=\linewidth]{plots/basic_autocorrelation.png}
    \caption{Autocorrelation of the simulation grouped by session}
    \label{fig:basic_autocorr_per_session}
\end{figure}


% Fat-tailed returns
The distribution of the returns are plotted in ~\ref{fig:basic_return_distr}. As observed in the plot, the fat-tailed returns
clearly exist in the intrasession period but not necessarily per session basis. For intrasession returns, there is a clear spike in the mean of the 
distribution and the tails spread out but these are not visible in the returns between session. In addition, the per-session distribution
is not smooth even though the sample consists of 22 million trades. The cause for this may be that the prices are not 
continuous because of the tick size: the minimum possible amount for the price to go up or down is one. Furthermore, prices are
stationary thus the returns are produced by a specific set of prices. Therefore, the returns are also inherently discrete. One option
to mitigate this would be to increase the amount of currency to increase the price and increase the absolute
variation of the prices. Jarque-Bera goodness of fit test suggests that both time periods are not normally distributed, %TODO Supervisor: Onko tama esitelty teoriassa? Kerro mika se on. "Jarque-Bera goodness of fit test suggests"
as the p-values are practically zero in both cases thus the null hypothesis is rejected. The null hypothesis of the Jarque-Bera test 
is a joint hypothesis that the distribution has no skewness and no excess kurtosis. However, as discussed, the distributions are not 
convincingly continuous thus it is unclear how robust these results are. The kurtosis of the distribution for intrasession is 5.0 
and -0.5 for session basis and therefore this indicates that the intrasession returns are leptokurtic while the returns between 
sessions are slightly lighter in tails than a normal distribution. To summarize, the model does have fatter tails than a normal distribution
for intrasession returns but not between sessions. 

% \citet{Raberto05} found fat tails in intra day

\begin{figure}[H]
    \centering
    \begin{subfigure}{.5\textwidth}
        \centering
        \includegraphics[width=\linewidth]{plots/basic_fat_tails_per_session.png}
        \caption{Between last prices of sessions}
        \label{fig:basic_return_distr_per_session}
    \end{subfigure}%
    \begin{subfigure}{.5\textwidth}
        \centering
        \includegraphics[width=\linewidth]{plots/basic_fat_tails_intrasession.png}
        \caption{Between trades}
        \label{fig:basic_return_distr_intrasession}
    \end{subfigure}
    \caption{Distributions of rate of returns}
    \label{fig:basic_return_distr}
\end{figure}

The spike in the distribution of the intrasession returns may be caused by the structure of the order book. If the order book 
sides are steep, as they are and illustrated in ~\ref{fig:basic_orderbook_evo}, it requires substantially sized 
order to push the opposing side back and have a significant effect on the price. Therefore, the price changes between trades 
are generally rather small until there comes an order that can buy or sell a significant chunk of the opposing side and have 
instant impact on the price. In such a case the opposing side may be weakened so much that the price change can be radical. 
Therefore, the reason for the spike in the returns may be that small and medium-sized orders may have a similar impact on the 
price but big orders may push the side relatively much further as the density of the order quantity in terms of price drops 
when moving further away from the last market price.

% Volatility clusters
The autocorrelation of the volatilities in figure ~\ref{fig:basic_volaclusters}
indicate there are some volatility clusters in short term. The figure was conducted by
dividing the trade prices to fixed length windows in which the volatilities were calculated per-window 
basis. As ACF plot is slowly decaying, the autocorrelation follows AR process. The market 
microstructure is able to produce volatility clusters without additional volatility feedback mechanics.
Furthermore, the order book evolution may play a role in creating the volatility clusters.

\begin{figure}[H]
    \includegraphics[width=\linewidth]{plots/basic_volaclusters_intra.png}
    \caption{Volatility Clusters}
    \label{fig:basic_volaclusters}
\end{figure}

These findings are somewhat in line with previous literature. \citet{Genoa01}, \citet{Raberto05} and \citet{LIU20082535} 
discussed that the microstructure is the cause for the fat-tailed returns. This is probably also the cause for the
fat-tails of intrasession returns observed in this thesis. However, the autocorrelation is different than what \citet{LIU20082535} 
observed with high call market-clearing frequencies. Regardless, the autocorrelations of returns observed in this thesis and their model
are both short term. Even though a call market has different mechanics than a continuous market, a call market with
very high clearing frequency behaves similarly as a continuous market does. \citet{Genoa01} and \citet{Raberto05} only discussed
forming volatility clusters in the context of their model which had volatility feedback mechanism thus it is unclear
whether the volatility clusters observed in the model created in this thesis are to be expected.
% TODO: how does the autocorrelation differ from LIU20082535?

\subsection{Shock injection}

To gain a better understanding of the order book dynamics of the model the next set of experiments are about injecting shocks 
to the order book. This is done by changing the weights the trading agents use to pick the type of an order they will submit. 
The experiment is started in equilibrium and then the probability of picking ask order is increased from 33\% to 43\% 
whereas the probability for bid order is reduced from 33\% to 23\%. The probability for inaction is kept at 33\%. 
After reaching a new equilibrium the weights are set back to 33\% for each option. Then the same experiment is repeated 
but instead of a negative shock, a positive shock is introduced using weights 43\%, 23\% and 33\% for bid order, ask order 
and inaction respectively. 

As this experiment is more exploratory in nature than the previous experiment the experiment may not need to be computationally
as heavy as the other experiments. Therefore, only 200 sessions are used to let the simulation reach equilibrium before and during 
the shocks. The number of trading agents is kept at 10 000 to reduce noise in interpreting the results. In addition, both experiments, 
introduction of negative and positive shocks, are conducted in the same simulation run.

The evolution of the order books in both shocks are presented in ~\ref{fig:shocks_orderbook}. Such shocks introduce a new equilibrium
price which is expectedly smaller in negative shock and higher in positive shock. When excess amount of bid orders flow to the market
there is a pressure for the price to increase and vice versa, which is observed here. 
% TODO: Explain when the shocks start and end in the graphs

As observed in the previous experiment, the converge to equilibrium occurs rapidly. Interestingly, the gradual convergence of price
to equilibrium is more prevalent in positive shock than in negative shock. In addition, the order book volumes on both sides of the
market are higher in negative shock than in positive shock, as observed in ~\ref{fig:market_depths_shocks}. Furthermore, the order 
book volumes are higher in negative shock than in the actual equilibrium. The reason for this is most likely that the smaller price 
allows the buyers to bid for higher quantities due to that they can afford more and vice versa. 

\begin{figure}[H]
    \includegraphics[width=\linewidth]{plots/shocks_order_book_evo.png}
    \caption{Order book evolution in shocks}
    \label{fig:shocks_orderbook}
\end{figure}

The comparable sizes of the order books can be better observed in ~\ref{fig:market_depths_shocks}. Initially, in negative shock,
there is a sudden surge of ask orders with sudden decline of bid orders. The existing bid orders are consumed by the ask orders 
until the price reaches to levels that the ask orders cannot push further due to the balance between the currency and the asset:
the buyers submit bids with higher quantities than previously as with the new price they can afford more. The positive shock has a similar 
effect: there is a sudden surge of bid orders and the ask side disappears for a moment and then gradually the equilibrium
price is reached as the ask side re-emerges. As demonstrated, the global balance between assets is not the only factor driving the 
market price and the imbalance of the sides of the submitted orders can drive the price also in zero-intelligent markets.

% TODO: There is surge of orders

\begin{figure}[H]
    \centering
    \begin{subfigure}{.5\textwidth}
      \centering
      \includegraphics[width=\linewidth]{plots/shocks_neg_market_depth_in_equilibrium.png}
      \caption{Market depth of negative shock}
      \label{fig:market_depth_neg_shock}
    \end{subfigure}%
    \begin{subfigure}{.5\textwidth}
      \centering
      \includegraphics[width=\linewidth]{plots/shocks_pos_market_depth_in_equilibrium.png}
      \caption{Market depth of positive shock}
      \label{fig:market_depth_pos_shock}
    \end{subfigure}
    \caption{Market depths of the shocks}
    \label{fig:market_depths_shocks}
\end{figure}


%   Lisaa tahan jonkinlainen "summary of results" kappale, jossa kerrot tarkeimmat loydokset kootusti 
%   --> ota materiaali conclusionista 
\subsection{Summary of Results}

% Equilibrium converge

The model converges rapidly to equilibrium price which was slightly lower
than expected from the ratio of the assets. Interestingly, the convergence
occurs more rapidly with lower starting price than higher. The model manages 
to somewhat capture the stylized facts observed in actual markets: the intrasession 
returns are fat-tailed and some short-term volatility clusters were observed. 
However, there was some negative moderately short-term autocorrelation in the 
absolute returns and the fat-tails of returns are not carried over to the returns between sessions. 
This is also somewhat in line with the literature with comparable models.
% TODO: Specify which studies

% Shocks
The model was also tested with negative and positive shocks by changing the 
traders' probabilities to submit bid and ask orders. It was found out that
the global balance between assets is not necessarily the only driving factor for the 
equilibrium price in a market populated with zero-intelligent traders 
and the price can be manipulated by changing these values. Furthermore, this finding suggests that 
solely the balance, or imbalance, of the sides in an order book may have an important role in 
driving the price. For example, if the bid side is thinner than the ask side it may be more probable 
that there will come an ask order that consumes a larger proportion of the bid side than the other way 
around. As the ask order consumes a large chunk of the bid side, the price will decrease rapidly 
or end up in free fall if the side is completely consumed. In addition, the negative shock
also increased the total quantities on both sides of the order book while
the positive shock had the opposite effect. Lower price is attributed to 
more liquid market even though the global balance of assets is unaffected.
This is due to that the amount of assets the buyers can bid is limited by the 
price while the amount of asset the sellers can sell are not. 