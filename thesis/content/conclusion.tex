

\section{Conclusion}

% Why this is unique
In this thesis, a generic computational market environment with simplistic
zero-intelligent trading agents was introduced. The model has several interesting features: 
the model supports multiple markets and any asset can act as the currency
for a market. In addition, tick sizes are implemented naturally by forcing
the quantities of the traded asset and the amount of currency to be integers.
The main purpose of the model was to discover how to build a computational artificial market 
in the perspective of how actual markets work and how to create such in a computational environment. 
The second purpose was to validate the created model and explore its dynamics.
In this chapter, the conclusions of the thesis are presented in form of answering to the 
research questions. Then the limitations of the model are discussed and some further topics 
of research are presented that emerged during the creation of the model.


\subsection{Answering to the Research Questions}

The first research question of this thesis was the following:
\begin{quote}
\textit{What are the aspects needed to be taken into consideration in designing of a computational 
artificial market according to the existing scientific literature?}
\end{quote}
Designing a computational artificial market is partly a technical and partly a domain-specific 
problem. Even though the focus of such models is to replicate the mechanics and the dynamics 
of real markets a proper technical design should not be disregarded. The software architecture 
is the foundation of the model and defines the maximum complexity achievable with the model. 
On the other hand, the purpose of artificial markets is to answer questions related to 
finance and economics thus understanding the mechanics of actual markets is also crucial. 

There are several choices to be made in terms of the domain of finance 
in designing of an artificial market:
\begin{itemize}
    \item Market organization: organized or unorganized,
	\item Execution system: order-driven or quote-driven,
	\item Clearing cycle: continuous or call-market,
    \item Rationality of the traders: zero-intelligent, behavioural or intelligent,
    \item Budget constrain: whether to allow short selling and borrowing, 
    \item Number of markets: whether to implement a multi-market ecosystem, and 
    \item Tick size: whether to implement minimum allowed price increment and decrement
\end{itemize}

There are also several technical challenges to be addressed in the creation 
of a computational market:
\begin{itemize}
    \item What are the required components to capture the dynamics modelled?
	\item How the market, traders and other components interact with each other?
	\item What algorithm to use to clear the market?
	\item How to avoid wealth leakage due to rounding errors?
	\item How to handle non-trading and what is the definition of time?
	\item What programming language to use?
	\item How the simulation process is orchestrated?
\end{itemize}

To answer the second research question:
\begin{quote}
    \textit{How to program a computational artificial market be constructed that is both abstract and generic?} 
\end{quote}

% The structure of an ASM
As artificial markets are experimental, they require many design decisions that may be harder to 
validate. It was found out to be convenient to separate the more generalized functionalities of the markets and 
operations of trading agents from the concrete implementation of the components. As a result, the 
model ended up to be more maintainable and easier to tune. In addition, as such models are stateful systems
a programming language that is able to produce stateful patterns similar to the object-oriented may 
be more useful than strictly functional language. With the implementations of mutable structs and parametrized 
functions, Julia language appeared to be a proper choice for such a system.

At a minimum, amount of components a computational market requires is two: the trading agents and the 
market. In order to keep the interaction between the traders and the markets simple, it may be beneficial to 
consider orders as a separate component with its own interface. Furthermore, having an additional component for 
assets enables the support for interest and dividend payments. The traders of an ASM are responsible for coming up with 
placement decisions and the markets are responsible for processing these orders and form trades accordingly. 

% Asset handling
In terms of asset handling, equal treatment of 
all assets should be considered if multi-asset or multi-market simulations are of interest. 
This requires an order reservation mechanism to be in place to prevent a breach in the 
budget constrain. In addition, the implementation of tick sizes turned out to be non-trivial. 
Modellers of artificial markets should be aware of the problems regarding the price inefficiency 
related to tick sizes when the market price drops low. 

% Time handling
%The implementation of time is also non-trivial. It may not be obvious how much time a trading session or 
%even the time between trades should correspond in terms of real time. In addition, the non-trading is also 
%an issue in artificial markets and should not be disregarded. Definition of time is important in order to 
%properly compare the results of the model to the phenomena in real markets.

To answer the third research question:
\begin{quote}
    \textit{How does the market price behave in a computational artificial market with continuous
    double auction as the market microstructure and zero-intelligent traders as the 
    trading agents?} 
\end{quote}
The experiments conducted with this model indicate similar results as the observations from 
the previous literature: the market price of such a market is efficient in unchanging market conditions
and rapidly converges to the equilibrium. The price dynamics have some similarities to the dynamics 
observed in the real market such as medium to long term fat-tails of return, volatility clusters and 
lack of autocorrelation. However, these properties do not hold in shorter time periods. In addition, 
the structure of the order book was observed to be an important factor in driving the market price and 
not just the global balance of assets. 


\subsection{Limitations of the Model}

% Time handling: unclear how one timestamp translates to real time (ie. minutes)
% issue with bid-ask symmetricity
% assuptions with ZI traders
% Lack of parallerization: some simulations took like 6 hours
% Should the ZI traders distribution's mean be linked to best price of the opposite book instead or something else?

% Lack of real data behind the model & time handling
There was no real order book data to reference in the creation of the model
introduced in this thesis. Therefore, it is unclear how aspects of this
model translate to actual markets or what the characteristics of the 
asset could be that this model resembles. For instance, the time aspect
remains unknown as it is not obvious what the time interval could
realistically be between the placement decisions or the trading 
sessions. Furthermore, the definition of short-term and long-term is
relatively unclear in the model.

% Tick size, integer quantity, bid-ask asymmetricity
Even though introducing
tick sizes ultimately made the model more realistic it did cause a 
set of challenges. With a combination of the traded quantities being
also integers the model behaves inefficiently with very low and very
high market prices due to the quantity and the price increments. This
is however also present in actual markets but there are solutions to handle
this such as splits or reverse splits in stock markets. Perhaps an
automated splitting mechanics could overcome this issue.

% ZI investor limitations
Even though the sophistication of the trading agents was in little
importance for the purpose of this thesis, there were some design
choices that may have had an impact on the dynamics of the model.
The definition of zero intelligence is not explicit in terms of
concrete behaviour. Whether it is less intelligent to draw the
order prices from a normal distribution than from a uniform
distribution or not remains debatable. In addition, both choices
introduce parameters to be defined. Setting the mean of the
normal distribution of the order prices to float with the 
market price possibly enabled some of the interesting dynamics
observed in the simulation but it may have taken part in causing the
autocorrelation in the model. Whether it is more appropriate
to fix the price to other values derived from the market,
such as the best price of the opposite book as done by
\citet{Genoa01}, also remains unclear. In addition, the 
standard deviation of the distribution was set to a constant
which could also be a factor playing a role in the dynamics 
of the order book with different equilibrium prices. Setting
it to be dependent on the previous volatility of the model
could produce interesting dynamics including the volatility
cluster but this could be seen as a violation of the definition
of zero intelligence as they would then be able to react on the risk level
of the market. Setting it to depend on the absolute 
current price level of the market may be more appropriate to 
maintain the level of rationality of the investors. 

% Technical aspects
Regarding the technical aspects of the model, some experiments 
took hours to complete. The model is thoroughly parametrized but 
there could be a substantial increase to code execution to be 
achieved with better type stability, by avoiding the use of 
abstract types and parallelizing some sections of the code. 
Optimizing may however increase the complexity of the code 
causing it harder to debug and maintain. The code did, however, 
run fast enough for the experiments of this thesis but may be 
an issue if complex and long multi-asset simulations are tested. 

% Language choice
Julia programming language suited to the task well. The functional nature
of the language provided a straightforward way to implement matching engine
efficiently. In addition, the mutable structs enabled modularization of the
components of the model. However, as the language is relatively new it is
lacking extensive ecosystems as more matured languages, such as Python,
has. This was not a problem in the development of this model as the topic is 
rather specific and does not require complex algorithms. This may come as an
issue if the complexity of the trader agents is increased. \citet{JuliaML2020} 
discussed the recent state of the language, particularly in terms of machine 
learning, and they noted that the language is more mature in the area of
supervised learning algorithms than in the area of unsupervised algorithms.
They did not, however, discuss evolutionary algorithms or reinforcement
learning which are more prevalent in the area of agent-based computational economics.
According to them the greatest challenge of the language is the small ecosystem
of packages compared to other high-level languages. This can be mitigated
with interfacing other languages that do have the required packages but 
\author{JuliaML2020} also argued that there is room for improvement for embedding
other languages to Julia and vice versa.

\subsection{Further Research Topics}
% Divident/interest payout
% Only simple stuff done in this thesis
There are several interesting additional experiments that could be done with
the model out-of-the-box. The parameters of the zero-intelligent traders could
be experimented with or the model could be used to study the behaviour of 
multi-asset markets including:
\begin{itemize}
    \item cross asset markets: a set of markets \emph{X}, \emph{Y} and \emph{Z} 
        in which \emph{X} trades asset \emph{b} with asset \emph{a}, 
        \emph{Y} trades asset \emph{c} with asset \emph{b} 
        and market \emph{Z} trades asset \emph{d} with asset \emph{c},
    \item arbitrage enabled markets: a set of markets \emph{X} and \emph{Y} 
        in which both markets trade asset \emph{a} with asset \emph{b},
    \item opposing markets: a set of markets \emph{X} and \emph{Y} 
        in which \emph{X} trades asset \emph{b} with asset \emph{a} 
        and \emph{Y} trades asset \emph{a} with asset \emph{b}
\end{itemize}

% More advanded traders
Also, as the project is heavily modularized the structure allows for the development of more advanced
trading agents. The model could be transformed to experiment with more advanced
market setups introduced by the literature of agent-based computational economics. As discussed,
there are some packages for machine learning found in Julia's ecosystem thus 
a learning trading agent could be constructed with readily available algorithms.

% More advanced microstructure
As it is relatively trivial to modify the clearing algorithm of the model to 
act as a call market one could also build more a precise representation of a
specific stock exchange. With the introduction of opening and closing crosses,
which are essentially call markets at the beginning and at the end of each 
trading day, a typical Nasdaq's trading day could be imitated. In addition,
the exclusion of pre- and post-market could be simulated by allowing only
specific traders to trade in specific trading sessions. This could be done
by modifying the main level of the simulation. 

% Centralized market place
As the effect of the microstructure to the market dynamics is relatively 
less understood in decentralized markets it would be interesting to modify 
the model to such and run similar analysis. Instead of exchanges, the trading 
would be conducted via trading networks: each trader has links to other traders, 
trading partners. When a trader states an order its trading partners are 
looped through and checked if any of the partners is willing to participate 
in the order and form a trade. However, such modification would require 
completely new matching engine and the trading agents need to be substantially 
refactored.
