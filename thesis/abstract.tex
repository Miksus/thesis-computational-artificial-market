\thispagestyle{plain}
\begin{center}
    \Large
    \textbf{Building a Computational Artificial Market}
        
    \vspace{0.4cm}
    %\large
        
    \vspace{0.4cm}
    
    \begin{flushleft}
        \normalsize
        \textbf{Author:} Mikael Koli \\
        \textbf{Major:} Strategic Finance and Business Analytics \\
        \textbf{Faculty:} School of Business and Management \\
        \textbf{Master\char39s Thesis:} Lappeenranta University of Technology \\
        \textbf{Year:} 2020 \\
        \textbf{Examiners:} Postdoctoral researcher Jyrki Savolainen, Professor Mikael Collan
    \end{flushleft}
        
    \vspace{0.5cm}
    \textbf{Abstract}
\end{center}

% TODO: Purpose of the thesis is also to act as road map to build such a model
Computational economics is a branch of experimental economics that aims to 
recreate and explain the dynamics of real markets using computational simulations. Researchers of
empirical economics have often very limited ability to control the market environment 
to study specific phenomena but such restrictions do not apply in pure computational
environments. The purpose of this thesis is to build a generic computational market. As the
previous literature is lacking in thorough examples of how to build such a model, the mechanics of this
model are discussed in detail. 

The microstructure of the model is a double auction or a limit order book market with continuous 
clearing. The trading agents of the model are budget constrained and zero-intelligent. 
The model supports multi-asset simulations and the meaning of currency is 
abstracted and any asset in the model can act as the currency or the traded asset for a market. 
The representativeness of the model is validated using stylized facts and the behaviour
of the order book is studied with various experiments.

The model produced similar results as the similar models from previous literature: zero-intelligent
traders do converge efficiently to equilibrium and some of the stylized facts can be produced
with such a simplistic model. As a contribution to the existing literature, the testing of the model 
revealed several additional interesting observations such as that the global balance between assets in 
zero-intelligent markets is not the only factor driving the equilibrium price: also the probability 
of issuing bid and ask orders affect the equilibrium. The structure of the order book has itself 
an effect on the equilibrium price.

\textbf{Keywords:} Computational Economics, Experimental Economics, Artificial Stock Market