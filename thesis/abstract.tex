\thispagestyle{plain}
\begin{center}
    \Large
    \textbf{Building a Computational Artificial Market}
        
    \vspace{0.4cm}
    \large
        
    \vspace{0.4cm}
    \textbf{Mikael Koli} \\
    July 10, 2020 % TODO
       
    \vspace{0.9cm}
    \textbf{Abstract}
\end{center}

% TODO: Purpose of the thesis is also to act as road map to build such a model
Computational economics is a branch of experimental economics that aims to 
recreate and explain dynamics of real markets using computational simulations. Researchers of
empirical economics has often very limited ability to control the market environment 
in order to studyspecific phenomena but such restrictions do not apply in pure computational
environments. The purpose of this thesis is to build a generic computational market. As the
previous literature is lacking on thorough examples of how to build such model the mechanics of this
model are discussed with detail. The model is intended to be as flexible and abstracted as possible 
while keeping the complexity of the model at minimum. The trading agents are budget constrained 
and zero intelligent in nature and the market can be described as a continuous double auction or a 
limit order book market. The model supports multi-asset simulations and the meaning of currency is 
abstracted away and any asset in the model can act as the currency or the traded asset for a market. 
The representativeness of the model is validated using stylized facts and the behaviour
of the order book is studied with various experiments.

The model produced similar results as the similar models from previous literature: zero-intelligent
traders do converge efficiently to an equilibrium and some of the stylized facts can be produced
with such simplistic model. Testing the model revealed several additional interesting observations.
The bid and ask sides of a generic artificial market with tick sizes are asymmetric by nature: the 
price of the bid orders are limited with the amount of currency able to be allocated to the order 
but such limitation is not prevalent with ask orders. Also, the global balance between assets in 
zero-intelligent markets is not the only factor driving the equilibrium price: also the probability 
of issuing bid and ask orders affect to the equilibrium.

Keywords: Computational Economics, Experimental Economics, Artificial Stock Market